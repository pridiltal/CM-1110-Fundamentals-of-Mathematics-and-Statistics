\documentclass[]{book}
\usepackage{lmodern}
\usepackage{amssymb,amsmath}
\usepackage{ifxetex,ifluatex}
\usepackage{fixltx2e} % provides \textsubscript
\ifnum 0\ifxetex 1\fi\ifluatex 1\fi=0 % if pdftex
  \usepackage[T1]{fontenc}
  \usepackage[utf8]{inputenc}
\else % if luatex or xelatex
  \ifxetex
    \usepackage{mathspec}
  \else
    \usepackage{fontspec}
  \fi
  \defaultfontfeatures{Ligatures=TeX,Scale=MatchLowercase}
\fi
% use upquote if available, for straight quotes in verbatim environments
\IfFileExists{upquote.sty}{\usepackage{upquote}}{}
% use microtype if available
\IfFileExists{microtype.sty}{%
\usepackage{microtype}
\UseMicrotypeSet[protrusion]{basicmath} % disable protrusion for tt fonts
}{}
\usepackage{hyperref}
\hypersetup{unicode=true,
            pdftitle={CM 1110 Fundamentals of Mathematics and Statistics},
            pdfauthor={Dr.~Priyanga D. Talagala},
            pdfborder={0 0 0},
            breaklinks=true}
\urlstyle{same}  % don't use monospace font for urls
\usepackage{natbib}
\bibliographystyle{apalike}
\usepackage{longtable,booktabs}
\usepackage{graphicx,grffile}
\makeatletter
\def\maxwidth{\ifdim\Gin@nat@width>\linewidth\linewidth\else\Gin@nat@width\fi}
\def\maxheight{\ifdim\Gin@nat@height>\textheight\textheight\else\Gin@nat@height\fi}
\makeatother
% Scale images if necessary, so that they will not overflow the page
% margins by default, and it is still possible to overwrite the defaults
% using explicit options in \includegraphics[width, height, ...]{}
\setkeys{Gin}{width=\maxwidth,height=\maxheight,keepaspectratio}
\IfFileExists{parskip.sty}{%
\usepackage{parskip}
}{% else
\setlength{\parindent}{0pt}
\setlength{\parskip}{6pt plus 2pt minus 1pt}
}
\setlength{\emergencystretch}{3em}  % prevent overfull lines
\providecommand{\tightlist}{%
  \setlength{\itemsep}{0pt}\setlength{\parskip}{0pt}}
\setcounter{secnumdepth}{5}
% Redefines (sub)paragraphs to behave more like sections
\ifx\paragraph\undefined\else
\let\oldparagraph\paragraph
\renewcommand{\paragraph}[1]{\oldparagraph{#1}\mbox{}}
\fi
\ifx\subparagraph\undefined\else
\let\oldsubparagraph\subparagraph
\renewcommand{\subparagraph}[1]{\oldsubparagraph{#1}\mbox{}}
\fi

%%% Use protect on footnotes to avoid problems with footnotes in titles
\let\rmarkdownfootnote\footnote%
\def\footnote{\protect\rmarkdownfootnote}

%%% Change title format to be more compact
\usepackage{titling}

% Create subtitle command for use in maketitle
\providecommand{\subtitle}[1]{
  \posttitle{
    \begin{center}\large#1\end{center}
    }
}

\setlength{\droptitle}{-2em}

  \title{CM 1110 Fundamentals of Mathematics and Statistics}
    \pretitle{\vspace{\droptitle}\centering\huge}
  \posttitle{\par}
    \author{Dr.~Priyanga D. Talagala}
    \preauthor{\centering\large\emph}
  \postauthor{\par}
      \predate{\centering\large\emph}
  \postdate{\par}
    \date{2020-02-02}

\usepackage{booktabs}
\usepackage{amsthm}
\makeatletter
\def\thm@space@setup{%
  \thm@preskip=8pt plus 2pt minus 4pt
  \thm@postskip=\thm@preskip
}
\makeatother

\begin{document}
\maketitle

{
\setcounter{tocdepth}{1}
\tableofcontents
}
\hypertarget{course-syllabus}{%
\chapter*{Course Syllabus}\label{course-syllabus}}
\addcontentsline{toc}{chapter}{Course Syllabus}

\hypertarget{pre-requisites}{%
\section*{Pre-requisites}\label{pre-requisites}}
\addcontentsline{toc}{section}{Pre-requisites}

None

\hypertarget{learning-outcomes}{%
\section*{Learning Outcomes}\label{learning-outcomes}}
\addcontentsline{toc}{section}{Learning Outcomes}

On successful completion of this module, students will be able to apply fundamental concepts in Mathematics and Statistics for real world problem solving.

\hypertarget{outline-syllabus}{%
\section*{Outline Syllabus}\label{outline-syllabus}}
\addcontentsline{toc}{section}{Outline Syllabus}

\begin{itemize}
\tightlist
\item
  Number Systems
\item
  Sequence and Series
\item
  Introduction to Logic
\item
  Boolean Algebra
\item
  Differentiation and Integration
\item
  Descriptive Statistics
\item
  Sets and Relations
\item
  Probability
\item
  Correlation and Regression
\end{itemize}

\hypertarget{method-of-assessment}{%
\section*{Method of Assessment}\label{method-of-assessment}}
\addcontentsline{toc}{section}{Method of Assessment}

\begin{itemize}
\tightlist
\item
  Mid-semester examination
\item
  End-semester examination
\end{itemize}

\hypertarget{lecturer}{%
\section*{Lecturer}\label{lecturer}}
\addcontentsline{toc}{section}{Lecturer}

Dr.~Priyanga Dilini Talagala

\hypertarget{schedule}{%
\section*{Schedule}\label{schedule}}
\addcontentsline{toc}{section}{Schedule}

Lectures: TBA

Consultation times: TBA

\hypertarget{number-systems}{%
\chapter{Number Systems}\label{number-systems}}

\hypertarget{sequence-and-series}{%
\chapter{Sequence and Series}\label{sequence-and-series}}

\hypertarget{introduction-to-logic}{%
\chapter{Introduction to Logic}\label{introduction-to-logic}}

\hypertarget{boolean-algebra}{%
\chapter{Boolean Algebra}\label{boolean-algebra}}

\hypertarget{differentiation-and-integration}{%
\chapter{Differentiation and Integration}\label{differentiation-and-integration}}

\hypertarget{descriptive-statistics}{%
\chapter{Descriptive Statistics}\label{descriptive-statistics}}

\pagenumbering{arabic}

\hypertarget{introduction-to-statistics}{%
\section{Introduction to Statistics}\label{introduction-to-statistics}}

\hypertarget{some-basic-terminologies-used-in-statistics}{%
\subsection{Some Basic Terminologies Used in Statistics}\label{some-basic-terminologies-used-in-statistics}}

\textbf{i Population}

\begin{itemize}
\tightlist
\item
  The set of \textbf{all} possible elements in the universe of interest to the researcher
\end{itemize}

\textbf{ii Sample}

\begin{itemize}
\tightlist
\item
  A Sample is a \textbf{subset} (a portion or part) of the population of interest
\item
  The sample must be a representative of the population of interest
\end{itemize}

\textbf{iii Element}

\begin{itemize}
\tightlist
\item
  Element is an \textbf{entity or object} which the information is collected.
\item
  \emph{Eg: Student, household, farm, company, tomato plant}
\end{itemize}

\textbf{iv Variable}

\begin{itemize}
\tightlist
\item
  A variable is \textbf{a feature characteristic which has different `values' or categories for different elements} (items/subjects/individuals)
\item
  \emph{Eg: Gender of client, brand of mobile phones, risk level, number of emails received per day, age of client, income of client}
\end{itemize}

\textbf{v Data}

\begin{itemize}
\item
  Data are \textbf{measurements or facts} that are collected from a statistical unit/entity of interest
\item
  We collect data on variables
\item
  Data are raw numbers or facts that must be processed (analysed) to get useful information.
\item
  We get information from data.
\item
  \emph{Eg:}
\end{itemize}

\textbf{\emph{Variable:}} \emph{Age (in years) of client}

\textbf{\emph{Data:}} \emph{21, 45, 18, 32, 30, 22, 23, 27}

\textbf{\emph{Information:}}

\emph{The mean age is 27.25 years}

\emph{The minimum age is 18 years}

\emph{The range of ages is 18-45}

\emph{The percentage of clients below 25 years of age: 50\%}

\textbf{vi Statistic}

\begin{itemize}
\tightlist
\item
  \textbf{Characteristic} of a \textbf{sample}
\item
  The value which calculated based on sample data
\end{itemize}

\textbf{vii Parameter}

\begin{itemize}
\tightlist
\item
  \textbf{Characteristic} of a \textbf{population}
\item
  The value which calculated based on population data
\end{itemize}

\textbf{viii Census}

\begin{itemize}
\tightlist
\item
  When a researcher \textbf{gathers data from the whole population for a given measurement,} it is called a census
\end{itemize}

\textbf{ix Sampling}

\begin{itemize}
\tightlist
\item
  When a researcher \textbf{gathers data from a sample of the population for a given measurement,} it is called sampling
\item
  The process of selecting a sample is also called sampling
\end{itemize}

\textbf{Why take a sample instead of studying every member of the population ?}

\begin{itemize}
\tightlist
\item
  Prohibitive cost of census
\item
  Destruction of item being studied may be required
\item
  Not possible to test or inspect all members of a population being studied.
\end{itemize}

\hypertarget{branches-of-statistics}{%
\subsection{Branches of Statistics}\label{branches-of-statistics}}

\begin{figure}

{\centering \includegraphics[width=1\linewidth]{figure/box1-1} 

}

\end{figure}

\textbf{i Descriptive Statistics}

\begin{itemize}
\tightlist
\item
  Descriptive statistics consists of organizing, summarizing and presenting data in an informative way.
\item
  The main purpose of descriptive statistics is to provide an overview of the data collected.
\item
  Descriptive statistics describes the data collected through frequency tables, graphs and summary measures (mean, variance, quartiles, etc.).
\end{itemize}

\textbf{ii Inferential Statistics}

\begin{itemize}
\tightlist
\item
  In inferential statistics sample data are used to draw inferences (i.e.~derive conclusions) or make predictions about the populations from which the sample has been taken.
\item
  This includes methods used to make decisions, estimates, predictions or generalizations about a population based on a sample.
\item
  This includes point estimations, interval estimation, test of hypotheses, regression analysis, time series analysis, multivariate analysis, etc.
\end{itemize}

\hypertarget{types-of-variables}{%
\subsection{Types of Variables}\label{types-of-variables}}

\begin{figure}

{\centering \includegraphics[width=1\linewidth]{figure/box2-1} 

}

\end{figure}

\hypertarget{qualitative-quantitative-variables}{%
\subsubsection{Qualitative / Quantitative Variables}\label{qualitative-quantitative-variables}}

\textbf{i Qualitative variable (Categorical variable)}

\begin{itemize}
\tightlist
\item
  The characteristic is a quality.
\item
  The data are categories.
\item
  They cannot be given numerical values.
\item
  However, it may be given a numerical label
\item
  Qualitative variables are sometimes referred as categorical variables.
\item
  \emph{Eg:}
\end{itemize}

\emph{Gender:}

\emph{Age group:}

\emph{Education level:}

\emph{A/L stream:}

\emph{Degree type:}

\emph{Hair colour: }

\emph{FIT student batch:}

\emph{Undergraduate level:}

\emph{Grade that you can obtain for CM 1110/ CM1130}

\textbf{ii Quantitative variable }

\begin{itemize}
\tightlist
\item
  The characteristic is a quantity
\item
  The data are numbers
\item
  Quantitative data require numeric values that indicate how much or how many.
\item
  They are obtained by counting or measuring with some scale
\item
  \emph{Eg: }
\end{itemize}

\emph{Number of family members:}

\emph{Number of emails received per day:}

\emph{Weight of a student:}

\emph{Age:}

\emph{Credit balance in the SIM card:}

\emph{Time remaining in class:}

\emph{Temperature:}

\emph{Marks }

\hypertarget{discrete-continuous-variables}{%
\subsubsection{Discrete/ Continuous Variables}\label{discrete-continuous-variables}}

\begin{itemize}
\tightlist
\item
  Quantitative variables can be classified as either discrete or continuous.
\end{itemize}

\textbf{i Discrete Variables}

\begin{itemize}
\tightlist
\item
  Quantitative
\item
  Usually the data are obtained by counting
\item
  There are impossible values between any two possible values
\item
  \emph{Eg:}
\end{itemize}

\emph{Number of family members:}

\emph{Number of emails received per day:}

\textbf{ii Continuous Variables}

\begin{itemize}
\tightlist
\item
  Quantitative
\item
  Usually, the data are obtained by measuring with a scale
\item
  There are no impossible values between any two possible values.(any value between any two possible values is also a possible value)
\item
  i.e a continuous variable can take any value within a specified range.
\item
  \emph{Eg:}
\end{itemize}

\emph{Weight of a student:}

\emph{Age:}

\emph{Credit balance in the SIM card:}

\emph{Time remaining in class:}

\emph{Temperature:}

\emph{Marks}

\hypertarget{scales-of-measurements}{%
\subsection{Scales of Measurements}\label{scales-of-measurements}}

\begin{figure}

{\centering \includegraphics[width=1\linewidth]{figure/box3-1} 

}

\end{figure}

\begin{itemize}
\tightlist
\item
  There are four levels of measurements called, \textbf{nominal, ordinal, interval and ratio.}
\item
  Each levels has its own rules and restrictions
\item
  Different levels of measurement contains different amount of information with respect to whatever the data are measuring
\end{itemize}

\textbf{i Nominal Scale}

\begin{itemize}
\tightlist
\item
  Qualitative
\item
  No order or ranking in categories.
\item
  These categories have to be mutually exclusive, i.e.~it should not be possible to place an individual or object in more than one category
\item
  A name of a category can be substituted by a number, but it will be mere label and have no numerical meaning
\end{itemize}

\textbf{ii Ordinal Scale}

\begin{itemize}
\tightlist
\item
  Qualitative
\item
  Categories can be ordered or ranked
\item
  A name of a category can be substituted by a number, but such a sequence does not indicate absolute quantities.
\item
  Difference between any two numbers on the scale does not have a numerical meaningful.
\item
  It cannot be assumed that the differences between adjacent numbers on the scale are equal.
\end{itemize}

\textbf{iii Interval Scale}

\begin{itemize}
\tightlist
\item
  Quantitative
\item
  Data can be ordered or ranked
\item
  There is no absolute zero point. Zero is only an arbitrary point with which other values can compare
\item
  Difference between two numbers is a meaningful numerical value
\item
  Ration of two numbers is not a meaningful numerical value.
\end{itemize}

\textbf{iv Ratio Scale}

\begin{itemize}
\tightlist
\item
  Quantitative
\item
  Highest level of measurement
\item
  There exist an absolute zero point (It has a true zero point)
\item
  Ratio between different measurements is meaningful
\end{itemize}

\hypertarget{presentation-of-data}{%
\section{Presentation of Data}\label{presentation-of-data}}

The sinking of the Titanic is one of the most infamous shipwrecks in history.

On April 15, 1912, during her maiden voyage, the widely considered ``unsinkable'' RMS Titanic sank after colliding with an iceberg. Unfortunately, there weren't enough lifeboats for everyone onboard, resulting in the death of 1502 out of 2224 passengers and crew

\footnote{Data source: \url{https://www.kaggle.com/varimp/a-mostly-tidyverse-tour-of-the-titanic}}

Here's a quick summary of our variables:

\begin{longtable}[]{@{}ll@{}}
\toprule
\begin{minipage}[b]{0.49\columnwidth}\raggedright
Variable Name\strut
\end{minipage} & \begin{minipage}[b]{0.45\columnwidth}\raggedright
Description\strut
\end{minipage}\tabularnewline
\midrule
\endhead
\begin{minipage}[t]{0.49\columnwidth}\raggedright
PassengerID\strut
\end{minipage} & \begin{minipage}[t]{0.45\columnwidth}\raggedright
Passenger ID (just a row number, so obviously not useful for prediction)\strut
\end{minipage}\tabularnewline
\begin{minipage}[t]{0.49\columnwidth}\raggedright
Survived\strut
\end{minipage} & \begin{minipage}[t]{0.45\columnwidth}\raggedright
Survived (1) or died (0)\strut
\end{minipage}\tabularnewline
\begin{minipage}[t]{0.49\columnwidth}\raggedright
Pclass\strut
\end{minipage} & \begin{minipage}[t]{0.45\columnwidth}\raggedright
Passenger class (first, second or third)\strut
\end{minipage}\tabularnewline
\begin{minipage}[t]{0.49\columnwidth}\raggedright
Name\strut
\end{minipage} & \begin{minipage}[t]{0.45\columnwidth}\raggedright
Passenger name\strut
\end{minipage}\tabularnewline
\begin{minipage}[t]{0.49\columnwidth}\raggedright
Gender\strut
\end{minipage} & \begin{minipage}[t]{0.45\columnwidth}\raggedright
Passenger Gender\strut
\end{minipage}\tabularnewline
\begin{minipage}[t]{0.49\columnwidth}\raggedright
Age\strut
\end{minipage} & \begin{minipage}[t]{0.45\columnwidth}\raggedright
Passenger age\strut
\end{minipage}\tabularnewline
\begin{minipage}[t]{0.49\columnwidth}\raggedright
SibSp\strut
\end{minipage} & \begin{minipage}[t]{0.45\columnwidth}\raggedright
Number of siblings/spouses aboard\strut
\end{minipage}\tabularnewline
\begin{minipage}[t]{0.49\columnwidth}\raggedright
Parch\strut
\end{minipage} & \begin{minipage}[t]{0.45\columnwidth}\raggedright
Number of parents/children aboard\strut
\end{minipage}\tabularnewline
\begin{minipage}[t]{0.49\columnwidth}\raggedright
Ticket\strut
\end{minipage} & \begin{minipage}[t]{0.45\columnwidth}\raggedright
Ticket number\strut
\end{minipage}\tabularnewline
\begin{minipage}[t]{0.49\columnwidth}\raggedright
Fare\strut
\end{minipage} & \begin{minipage}[t]{0.45\columnwidth}\raggedright
Fare\strut
\end{minipage}\tabularnewline
\begin{minipage}[t]{0.49\columnwidth}\raggedright
Cabin\strut
\end{minipage} & \begin{minipage}[t]{0.45\columnwidth}\raggedright
Cabin\strut
\end{minipage}\tabularnewline
\begin{minipage}[t]{0.49\columnwidth}\raggedright
Embarked\strut
\end{minipage} & \begin{minipage}[t]{0.45\columnwidth}\raggedright
Port of embarkation (S = Southampton, C = Cherbourg, Q = Queenstown)\strut
\end{minipage}\tabularnewline
\bottomrule
\end{longtable}

\hypertarget{tabular-presentations-of-data}{%
\subsection{Tabular Presentations of Data}\label{tabular-presentations-of-data}}

\textbf{Raw Data}

\begin{itemize}
\tightlist
\item
  Raw data are collected data that have not been organized numerically
\item
  Eg: Passenger age
\end{itemize}

\begin{verbatim}
##   PassengerId Survived Pclass
## 1           1        0      3
## 2           2        1      1
## 3           3        1      3
## 4           4        1      1
## 5           5        0      3
## 6           6        0      3
##                                                  Name    Sex Age SibSp Parch
## 1                             Braund, Mr. Owen Harris   male  22     1     0
## 2 Cumings, Mrs. John Bradley (Florence Briggs Thayer) female  38     1     0
## 3                              Heikkinen, Miss. Laina female  26     0     0
## 4        Futrelle, Mrs. Jacques Heath (Lily May Peel) female  35     1     0
## 5                            Allen, Mr. William Henry   male  35     0     0
## 6                                    Moran, Mr. James   male  NA     0     0
##             Ticket    Fare Cabin Embarked
## 1        A/5 21171  7.2500              S
## 2         PC 17599 71.2833   C85        C
## 3 STON/O2. 3101282  7.9250              S
## 4           113803 53.1000  C123        S
## 5           373450  8.0500              S
## 6           330877  8.4583              Q
\end{verbatim}

\begin{verbatim}
##  [1] 22 38 26 35 35 NA 54  2 27 14  4 58 20 39 14 55  2 NA 31 NA 35 34 15 28  8
## [26] 38 NA 19 NA NA 40 NA NA 66 28 42 NA 21 18 14
\end{verbatim}

\textbf{An array}

\begin{itemize}
\tightlist
\item
  An array is an arrangement of raw numerical data in ascending or descending order of magnitude.
\item
  Eg: Passenger age
\end{itemize}

\begin{verbatim}
##  [1]  2  2  4  8 14 14 14 15 18 19 20 21 22 26 27 28 28 31 34 35 35 35 38 38 39
## [26] 40 42 54 55 58 66
\end{verbatim}

\textbf{Frequency Table (Frequency Distributions)}

\begin{itemize}
\tightlist
\item
  A frequency table (frequency distribution) is a listing of the values a variable takes in a data set, along with how often (frequently) each value occurs
\item
  frequency can be recorded as a

  \begin{itemize}
  \tightlist
  \item
    \textbf{frequency or count:} the number of times a value occurs, or
  \item
    \textbf{percentage frequency:} the percentage of times a value occurs
  \end{itemize}
\item
  Percentage frequency can be calculated as,
\end{itemize}

\[Percentage frequency = \frac{a}{b} \times100 \%\]

\begin{itemize}
\item
  The objective of constructing a frequency table are as follows

  \begin{itemize}
  \tightlist
  \item
    to organize the data in a meaningful manner
  \item
    to determine the nature or shape of the distribution
  \item
    to draw charts and graphs for the presentation of data
  \item
    to facilitate computational procedures for measures of average and spread
  \item
    to make comparisons between different data sets
  \end{itemize}
\item
  \begin{itemize}
  \tightlist
  \item
    There are two basic types of frequency tables

    \begin{enumerate}
    \def\labelenumi{\arabic{enumi}.}
    \tightlist
    \item
      Simple frequency tables (Ungrouped frequency distribution)
    \item
      Grouped frequency distribution
    \end{enumerate}
  \end{itemize}
\end{itemize}

\hypertarget{simple-frequency-table-ungrouped-frequency-distribution}{%
\subsubsection{Simple frequency table (Ungrouped frequency distribution)}\label{simple-frequency-table-ungrouped-frequency-distribution}}

\begin{itemize}
\tightlist
\item
  Each possible value or category is taken as a class
\item
  More suitable for

  \begin{itemize}
  \tightlist
  \item
    Qualitative variables
  \item
    Discrete variables
  \end{itemize}
\item
  Sometimes construct for continuous variables when there is a small number of possible values between the minimum and maximum.
\end{itemize}

Examples:

\textbf{CASE I:}

Example 1

The native countries of 56 students from a certain education institute are as follows:

\begin{verbatim}
##  [1] "SL" "BD" "SL" "SL" "SL" "SL" "IN" "SL" "SL" "SL" "BD" "SL" "SL" "SL" "IN"
## [16] "SL" "SL" "BD" "SL" "SL" "SL" "SL" "SL" "SL" "SL" "SL" "SL" "MD" "SL" "SL"
## [31] "SL" "SL" "SL" "SL" "PK" "MD" "PK" "SL" "SL" "SL" "SL" "SL" "PK" "MD" "SL"
## [46] "SL" "SL" "SL" "SL" "SL" "SL" "SL" "SL" "SL" "MD" "MD"
\end{verbatim}

BD- Bangladesh, IN-India, MD-Maldives, PK-Pakistan, SL- Sri Lanka

Construct a frequency table

\label{tab:unnamed-chunk-8}The frequency distribution of native countries

Native Country

Count

Percentage (\%)

Bangladesh

3

5.357

India

2

3.571

Maldives

5

8.929

Pakistan

3

5.357

Sri Lanka

43

76.786

Total

56

100.000

\textbf{CASE II:}

Example 2

The grades of 30 students for Statistics are as follows:

\begin{verbatim}
##  [1] "B" "C" "B" "D" "B" "C" "C" "A" "B" "C" "C" "B" "E" "B" "B" "D" "D" "F" "B"
## [20] "D" "D" "A" "B" "A" "B" "C" "E" "A" "A"
\end{verbatim}

Construct a frequency table

\label{tab:unnamed-chunk-10}The frequency distribution of grades for Statistics

Grade

Count

Percentage (\%)

A

5

17.241

B

10

34.483

C

6

20.690

D

5

17.241

E

2

6.897

F

1

3.448

Total

29

100.000

\textbf{CASE III:}

Example 3

The number of family members of a sample of undergraduates of Batch 19 are as follows:

\begin{verbatim}
##  [1] 7 5 3 4 5 4 3 6 4 4 5 2 7 4 5 6 4 4 3 5
\end{verbatim}

Construct a frequency table

\label{tab:unnamed-chunk-12}The frequency distribution of the number of family members

Number of family members

Count

Percentage (\%)

A

5

17.241

B

10

34.483

C

6

20.690

D

5

17.241

E

2

6.897

F

1

3.448

Total

29

100.000

\textbf{CASE IV:}

Example 4

The ages (in years) of a sample of undergraduates of Batch 19 are as follows:

\begin{verbatim}
##  [1] 21 22 22 23 22 24 24 23 21 22 23 22 22 23 21 21 22 23 22 23
\end{verbatim}

Construct a frequency table

\label{tab:unnamed-chunk-14}The frequency distribution of ages of undergraduates of Batch 19

Age (years)

Count

Percentage (\%)

A

5

17.241

B

10

34.483

C

6

20.690

D

5

17.241

E

2

6.897

F

1

3.448

Total

29

100.000

\hypertarget{grouped-frequency-distribution}{%
\subsubsection{Grouped frequency distribution}\label{grouped-frequency-distribution}}

\begin{itemize}
\tightlist
\item
  A grouped frequency distribution (table) is obtained by constructing classes (or intervals) for the data and then listing the corresponding number of values in each interval.
\item
  Suitable for quantitative variables with large number of possible values in the range of data.
\item
  Note that when items have been grouped in this way, their individual values are lost.
\item
  When studying about frequency distributions it is very important to know the meaning of the following terms
\end{itemize}

\textbf{i Class intervals}

\begin{itemize}
\tightlist
\item
  In a frequency distribution the total range of the observations are divided into a number of classes. Those are called \emph{class intervals}
\item
  Eg: Class intervals: 10-14, 15-19, 20-24, \ldots{}, 40-44
\end{itemize}

\textbf{ii Class limits}

\begin{itemize}
\tightlist
\item
  Class limits are the smallest and largest piece of data value that can fall into a given class.
\item
  In the class interval 10-14, the end numbers, 10 and 14, are called class limits
\item
  The smaller number (10) is the \emph{lower class limit}
\item
  The larger number (14) is the \emph{upper class limit}
\end{itemize}

\textbf{iii Class boundaries}

\begin{itemize}
\tightlist
\item
  Class boundaries are obtained by adding the upper limit of one class interval to the lower limit of the next-higher class interval and dividing by 2.
\item
  Class boundaries are also called \textbf{True class limits}
\item
  Class boundaries \textbf{should not} \emph{coincide with actual observations}
\end{itemize}

\begin{longtable}[]{@{}ll@{}}
\toprule
Class interval & Class boundaries\tabularnewline
\midrule
\endhead
10 - 14 & 9.5 -- 14.5\tabularnewline
15 - 19 & 14.5 -- 19.5\tabularnewline
20 - 24 & 19.5 -- 24.5\tabularnewline
25 - 29 & 24.5 -- 29.5\tabularnewline
30 - 34 & 29.5 -- 34.5\tabularnewline
35 - 39 & 34.5 -- 39.5\tabularnewline
40 - 44 & 39.5 -- 44.5\tabularnewline
\bottomrule
\end{longtable}

\textbf{iv The size or width of a class interval}

\begin{itemize}
\tightlist
\item
  The size or width of a class interval is the difference between the \emph{lower and upper class boundaries}
\item
  It is also referred to as the \emph{class width, class size, or class length}
\item
  Eg: The class width for the class 10-14 is = 14.5-9.5 = 5
\end{itemize}

\textbf{v The class mark ( Midpoint of the class)}

\begin{itemize}
\tightlist
\item
  Midpoint of the class
\item
  Also called as \emph{class midpoint}
\item
  \(\text{Midpoint of the class} = \frac{\text{Lower limit} + \text{Upper limit}}{2}\)
\end{itemize}

or

\begin{itemize}
\tightlist
\item
  \(\text{Midpoint of the class} = \frac{\text{Lower boundary} + \text{Upper boundary}}{2}\)
\end{itemize}

\textbf{vi Open class intervals}

\begin{itemize}
\item
  A class interval that, at least theoretically, has either no upper class limit or no lower class limit indicated is called an \emph{open class interval}
\item
  For example, referring to age groups of individuals, the class interval ``65 year and over'' is an open class interval
\end{itemize}

\textbf{Rules and Practices for constructing grouped frequency tables}

\begin{itemize}
\tightlist
\item
  Every data value should be in an interval
\item
  The intervals should be mutually exclusive
\item
  The classes of the distribution must be arrayed in size order.
\item
  The number of classes not less than 5 or not greater than 15 is recommended.
\item
  The following formula is often used to determine the number of classes:
  If n is the number of observations, then
\end{itemize}

\[\text{Number of classes} = \sqrt{n}\]

\[\text{Width of the class interval} = \frac{Range}{\sqrt{n}}= \frac{Min-Max}{\sqrt{n}}\]

\begin{itemize}
\tightlist
\item
  Data should be represented within classes having limits which the data can attain
\item
  Classes should be continuous
\item
  By convention, the beginning of the interval is given the appropriate exact value, rather than the end.\\
  Eg: intervals of 0-49, 50-99,100-149 would be preferred over the intervals 1-50, 51-100, 101-150 etc.
\item
  The number f observations falling into each category or class interval (class frequency) can be easily found using \emph{tally marks}.
\end{itemize}

Examples:

In a grouped frequency distribution, class intervals can be constructed in different ways

Example 1

\begin{longtable}[]{@{}ll@{}}
\toprule
Class interval & Number of students\tabularnewline
\midrule
\endhead
10 - 14 & 4\tabularnewline
15 - 19 & 5\tabularnewline
20 - 24 & 11\tabularnewline
25 - 29 & 9\tabularnewline
30 - 34 & 6\tabularnewline
35 - 39 & 3\tabularnewline
40 - 44 & 2\tabularnewline
\bottomrule
\end{longtable}

Example 2

\begin{longtable}[]{@{}ll@{}}
\toprule
Salary & Number of employees\tabularnewline
\midrule
\endhead
0 -- 1999 & 1\tabularnewline
2000 -- 3999 & 31\tabularnewline
4000 -- 5999 & 18\tabularnewline
6000 -- 7999 & 4\tabularnewline
8000 -- 9999 & 2\tabularnewline
10000 - 11999 & 1\tabularnewline
12000 -- 13999 & 0\tabularnewline
14000 -- 15999 & 0\tabularnewline
16000 -- 17999 & 1\tabularnewline
18000 -19999 & 1\tabularnewline
20000-21999 & 1\tabularnewline
\bottomrule
\end{longtable}

\begin{longtable}[]{@{}ll@{}}
\toprule
Salary & Number of employees\tabularnewline
\midrule
\endhead
0 -- 1999 & 1\tabularnewline
2000 -- 3999 & 31\tabularnewline
4000 -- 5999 & 18\tabularnewline
6000 -- 7999 & 4\tabularnewline
8000 -- 9999 & 2\tabularnewline
10000 - 15999 & 1\tabularnewline
16000 -- 21999 & 3\tabularnewline
Total & 60\tabularnewline
\bottomrule
\end{longtable}

\textbf{Example 3}

\begin{longtable}[]{@{}ll@{}}
\toprule
Salary & Number of employees\tabularnewline
\midrule
\endhead
Less than 2000 & 1\tabularnewline
2000 -- 2999 & 11\tabularnewline
3000 -- 3999 & 20\tabularnewline
4000 -- 5999 & 18\tabularnewline
6000 -- 9999 & 6\tabularnewline
Greater than or equal to 10000 & 4\tabularnewline
Total & 60\tabularnewline
\bottomrule
\end{longtable}

\hypertarget{two-way-frequency-table}{%
\subsubsection{Two-way frequency table}\label{two-way-frequency-table}}

\begin{itemize}
\tightlist
\item
  Cross tabulation, Cross classification table, Contingency table, Two-way table
\item
  Display the relationship between two or more qualitative variables (categorical variables (nominal or ordinal))
\end{itemize}

\begin{verbatim}
## # A tibble: 2 x 4
##   Survived First Second Third
##   <chr>    <dbl>  <dbl> <dbl>
## 1 died        80     97   372
## 2 Survived   136     87   119
\end{verbatim}

\begin{verbatim}
## # A tibble: 2 x 4
##   Survived First Second Third
##   <chr>    <dbl>  <dbl> <dbl>
## 1 died      0.37   0.53  0.76
## 2 Survived  0.63   0.47  0.24
\end{verbatim}

\hypertarget{graphic-presentations-of-data}{%
\subsection{Graphic Presentations of Data}\label{graphic-presentations-of-data}}

\begin{itemize}
\tightlist
\item
  A diagram is a visual form for presentation of statistical data.
\item
  The form of the diagram varies according to the nature of the data
\end{itemize}

\hypertarget{describing-qualitative-data}{%
\subsubsection{Describing Qualitative Data}\label{describing-qualitative-data}}

\begin{itemize}
\tightlist
\item
  Bar chart / Pie chart
\item
  Suitable for

  \begin{itemize}
  \tightlist
  \item
    Qualitative variables (nominal or ordinal)
  \item
    Discrete variables (when the number of bars or number of different values is small)
  \end{itemize}
\end{itemize}

\textbf{I Bar Chart}

\begin{itemize}
\tightlist
\item
  A bar graph uses bars to represent discrete categories of data
\item
  It can be drawn either on horizontal (more common) or vertical base
\item
  A rectangle of equal width is drawn for each category
\item
  The height (in vertical bar chart) or the length (in horizontal bar chart) of the rectangle is equal to the category's \textbf{frequency} or \textbf{percentage}.
\end{itemize}

\begin{figure}

{\centering \includegraphics[width=1\linewidth]{figure/box6-1} 

}

\end{figure}

\textbf{i Simple Bar Chart}

\begin{itemize}
\tightlist
\item
  Only one categorical variable can be presented
\item
  Often used in conjunction with simple frequency tables
\item
  The bars do not touch each other
\item
  The gaps between adjacent bars are same in length
\end{itemize}

\label{tab:unnamed-chunk-17}The frequency distribution of the Passenger class

Passenger class

Count

Percentage

First

216

24.242

Second

184

20.651

Third

491

55.107

\begin{center}\includegraphics[width=0.8\textwidth]{figure/passenger-1} \end{center}

\begin{figure}

{\centering \includegraphics[width=0.8\textwidth]{figure/passengerP -1} 

}

\caption{Composition of the passengers by passenger class (Sample size = 891)}(\#fig:passengerP )
\end{figure}

\textbf{ii Component Bar Chart }

\begin{itemize}
\tightlist
\item
  \textbf{Sub divided bar chart/ Stacked bar chart}
\item
  Use to compare two or more qualitative variables (nominal or ordinal)
\item
  Often used in conjunction with two way tables
\item
  Start by drawing a simple bar chart with the total figures.
\item
  The bars are then divided into the component parts
\item
  Can be drawn on absolute figures or percentages
\item
  The various components should be kept in the same order in each bar
\item
  To distinguish different components from one another, different colours or shades can be used
\end{itemize}

\begin{verbatim}
## # A tibble: 2 x 4
##   Survived First Second Third
##   <chr>    <dbl>  <dbl> <dbl>
## 1 died        80     97   372
## 2 Survived   136     87   119
\end{verbatim}

\begin{center}\includegraphics[width=0.8\textwidth]{figure/unnamed-chunk-19-1} \end{center}

**Percentage component bar chart
- When sub-divided bar chart is drawn on percentage basis it is called percentage bar chart
- The various components are expressed as percentage to the total
- All bars are equal in height

\hypertarget{sets-and-relations}{%
\chapter{Sets and Relations}\label{sets-and-relations}}

\hypertarget{probability}{%
\chapter{Probability}\label{probability}}

\hypertarget{correlation-and-regression}{%
\chapter{Correlation and Regression}\label{correlation-and-regression}}

\bibliography{book.bib,packages.bib}


\end{document}
