% Options for packages loaded elsewhere
\PassOptionsToPackage{unicode}{hyperref}
\PassOptionsToPackage{hyphens}{url}
%
\documentclass[
]{book}
\usepackage{lmodern}
\usepackage{amssymb,amsmath}
\usepackage{ifxetex,ifluatex}
\ifnum 0\ifxetex 1\fi\ifluatex 1\fi=0 % if pdftex
  \usepackage[T1]{fontenc}
  \usepackage[utf8]{inputenc}
  \usepackage{textcomp} % provide euro and other symbols
\else % if luatex or xetex
  \usepackage{unicode-math}
  \defaultfontfeatures{Scale=MatchLowercase}
  \defaultfontfeatures[\rmfamily]{Ligatures=TeX,Scale=1}
\fi
% Use upquote if available, for straight quotes in verbatim environments
\IfFileExists{upquote.sty}{\usepackage{upquote}}{}
\IfFileExists{microtype.sty}{% use microtype if available
  \usepackage[]{microtype}
  \UseMicrotypeSet[protrusion]{basicmath} % disable protrusion for tt fonts
}{}
\makeatletter
\@ifundefined{KOMAClassName}{% if non-KOMA class
  \IfFileExists{parskip.sty}{%
    \usepackage{parskip}
  }{% else
    \setlength{\parindent}{0pt}
    \setlength{\parskip}{6pt plus 2pt minus 1pt}}
}{% if KOMA class
  \KOMAoptions{parskip=half}}
\makeatother
\usepackage{xcolor}
\IfFileExists{xurl.sty}{\usepackage{xurl}}{} % add URL line breaks if available
\IfFileExists{bookmark.sty}{\usepackage{bookmark}}{\usepackage{hyperref}}
\hypersetup{
  pdftitle={CM 1110 Fundamentals of Mathematics and Statistics},
  pdfauthor={Dr.~Priyanga D. Talagala},
  hidelinks,
  pdfcreator={LaTeX via pandoc}}
\urlstyle{same} % disable monospaced font for URLs
\usepackage{longtable,booktabs}
% Correct order of tables after \paragraph or \subparagraph
\usepackage{etoolbox}
\makeatletter
\patchcmd\longtable{\par}{\if@noskipsec\mbox{}\fi\par}{}{}
\makeatother
% Allow footnotes in longtable head/foot
\IfFileExists{footnotehyper.sty}{\usepackage{footnotehyper}}{\usepackage{footnote}}
\makesavenoteenv{longtable}
\usepackage{graphicx,grffile}
\makeatletter
\def\maxwidth{\ifdim\Gin@nat@width>\linewidth\linewidth\else\Gin@nat@width\fi}
\def\maxheight{\ifdim\Gin@nat@height>\textheight\textheight\else\Gin@nat@height\fi}
\makeatother
% Scale images if necessary, so that they will not overflow the page
% margins by default, and it is still possible to overwrite the defaults
% using explicit options in \includegraphics[width, height, ...]{}
\setkeys{Gin}{width=\maxwidth,height=\maxheight,keepaspectratio}
% Set default figure placement to htbp
\makeatletter
\def\fps@figure{htbp}
\makeatother
\setlength{\emergencystretch}{3em} % prevent overfull lines
\providecommand{\tightlist}{%
  \setlength{\itemsep}{0pt}\setlength{\parskip}{0pt}}
\setcounter{secnumdepth}{5}
\usepackage{booktabs}
\usepackage{amsthm}
\makeatletter
\def\thm@space@setup{%
  \thm@preskip=8pt plus 2pt minus 4pt
  \thm@postskip=\thm@preskip
}
\makeatother
\usepackage{fancyhdr}
\pagestyle{fancy}
\fancyfoot[CO,CE]{Prepared by Dr. Priyanga D. Talagala}
\fancyfoot[LE,RO]{\thepage}
\usepackage[]{natbib}
\bibliographystyle{apalike}

\title{CM 1110 Fundamentals of Mathematics and Statistics}
\author{Dr.~Priyanga D. Talagala}
\date{2020-01-31}

\begin{document}
\maketitle

{
\setcounter{tocdepth}{1}
\tableofcontents
}
\hypertarget{course-syllabus}{%
\chapter*{Course Syllabus}\label{course-syllabus}}
\addcontentsline{toc}{chapter}{Course Syllabus}

\hypertarget{pre-requisites}{%
\section*{Pre-requisites}\label{pre-requisites}}
\addcontentsline{toc}{section}{Pre-requisites}

None

\hypertarget{learning-outcomes}{%
\section*{Learning Outcomes}\label{learning-outcomes}}
\addcontentsline{toc}{section}{Learning Outcomes}

On successful completion of this module, students will be able to apply fundamental concepts in Mathematics and Statistics for real world problem solving.

\hypertarget{outline-syllabus}{%
\section*{Outline Syllabus}\label{outline-syllabus}}
\addcontentsline{toc}{section}{Outline Syllabus}

\begin{itemize}
\tightlist
\item
  Number Systems
\item
  Sequence and Series
\item
  Introduction to Logic
\item
  Boolean Algebra
\item
  Differentiation and Integration
\item
  Descriptive Statistics
\item
  Sets and Relations
\item
  Probability
\item
  Correlation and Regression
\end{itemize}

\hypertarget{method-of-assessment}{%
\section*{Method of Assessment}\label{method-of-assessment}}
\addcontentsline{toc}{section}{Method of Assessment}

\begin{itemize}
\tightlist
\item
  Mid-semester examination
\item
  End-semester examination
\end{itemize}

\hypertarget{lecturer}{%
\section*{Lecturer}\label{lecturer}}
\addcontentsline{toc}{section}{Lecturer}

Dr.~Priyanga Dilini Talagala

\hypertarget{schedule}{%
\section*{Schedule}\label{schedule}}
\addcontentsline{toc}{section}{Schedule}

Lectures: TBA

Consultation times: TBA

\hypertarget{number-systems}{%
\chapter{Number Systems}\label{number-systems}}

\hypertarget{sequence-and-series}{%
\chapter{Sequence and Series}\label{sequence-and-series}}

\hypertarget{introduction-to-logic}{%
\chapter{Introduction to Logic}\label{introduction-to-logic}}

\hypertarget{boolean-algebra}{%
\chapter{Boolean Algebra}\label{boolean-algebra}}

\hypertarget{differentiation-and-integration}{%
\chapter{Differentiation and Integration}\label{differentiation-and-integration}}

\hypertarget{intro}{%
\chapter{Descriptive Statistics}\label{intro}}

\pagenumbering{arabic}

\hypertarget{introduction-to-statistics}{%
\section{Introduction to Statistics}\label{introduction-to-statistics}}

\hypertarget{some-basic-terminologies-used-in-statistics}{%
\subsection{Some Basic Terminologies Used in Statistics}\label{some-basic-terminologies-used-in-statistics}}

\textbf{i Population}

\begin{itemize}
\tightlist
\item
  The set of \textbf{all} possible elements in the universe of interest to the researcher
\end{itemize}

\textbf{ii Sample}

\begin{itemize}
\tightlist
\item
  A Sample is a \textbf{subset} (a portion or part) of the population of interest
\item
  The sample must be a representative of the population of interest
\end{itemize}

\textbf{iii Element}

\begin{itemize}
\tightlist
\item
  Element is an \textbf{entity or object} which the information is collected.
\item
  \emph{Eg: Student, household, farm, company, tomato plant}
\end{itemize}

\textbf{iv Variable}

\begin{itemize}
\tightlist
\item
  A variable is \textbf{a feature characteristic which has different `values' or categories for different elements} (items/subjects/individuals)
\item
  \emph{Eg: Gender of client, brand of mobile phones, risk level, number of emails received per day, age of client, income of client}
\end{itemize}

\textbf{v Data}

\begin{itemize}
\item
  Data are \textbf{measurements or facts} that are collected from a statistical unit/entity of interest
\item
  We collect data on variables
\item
  Data are raw numbers or facts that must be processed (analysed) to get useful information.
\item
  We get information from data.
\item
  \emph{Eg:}
\end{itemize}

\textbf{\emph{Variable:}} \emph{Age (in years) of client}

\textbf{\emph{Data:}} \emph{21, 45, 18, 32, 30, 22, 23, 27}

\textbf{\emph{Information:}}

\emph{The mean age is 27.25 years}

\emph{The minimum age is 18 years}

\emph{The range of ages is 18-45}

\emph{The percentage of clients below 25 years of age: 50\%}

\textbf{vi Statistic}

\begin{itemize}
\tightlist
\item
  \textbf{Characteristic} of a \textbf{sample}
\item
  The value which calculated based on sample data
\end{itemize}

\textbf{vii Parameter}

\begin{itemize}
\tightlist
\item
  \textbf{Characteristic} of a \textbf{population}
\item
  The value which calculated based on population data
\end{itemize}

\textbf{viii Census}

\begin{itemize}
\tightlist
\item
  When a researcher \textbf{gathers data from the whole population for a given measurement,} it is called a census
\end{itemize}

\textbf{ix Sampling}

\begin{itemize}
\tightlist
\item
  When a researcher \textbf{gathers data from a sample of the population for a given measurement,} it is called sampling
\item
  The process of selecting a sample is also called sampling
\end{itemize}

\textbf{Why take a sample instead of studying every member of the population ?}

\begin{itemize}
\tightlist
\item
  Prohibitive cost of census
\item
  Destruction of item being studied may be required
\item
  Not possible to test or inspect all members of a population being studied.
\end{itemize}

\hypertarget{branches-of-statistics}{%
\subsection{Branches of Statistics}\label{branches-of-statistics}}

\begin{figure}[h]

{\centering \includegraphics[width=1\linewidth]{figure/box1-1} 

}

\end{figure}

\textbf{i Descriptive Statistics}

\begin{itemize}
\tightlist
\item
  Descriptive statistics consists of organizing, summarizing and presenting data in an informative way.
\item
  The main purpose of descriptive statistics is to provide an overview of the data collected.
\item
  Descriptive statistics describes the data collected through frequency tables, graphs and summary measures (mean, variance, quartiles, etc.).
\end{itemize}

\textbf{ii Inferential Statistics}

\begin{itemize}
\tightlist
\item
  In inferential statistics sample data are used to draw inferences (i.e.~derive conclusions) or make predictions about the populations from which the sample has been taken.
\item
  This includes methods used to make decisions, estimates, predictions or generalizations about a population based on a sample.
\item
  This includes point estimations, interval estimation, test of hypotheses, regression analysis, time series analysis, multivariate analysis, etc.
\end{itemize}

\hypertarget{types-of-variables}{%
\subsection{Types of Variables}\label{types-of-variables}}

\begin{figure}[h]

{\centering \includegraphics[width=1\linewidth]{figure/box2-1} 

}

\end{figure}

\hypertarget{qualitative-quantitative-variables}{%
\subsubsection{Qualitative / Quantitative Variables}\label{qualitative-quantitative-variables}}

\textbf{i Qualitative variable (Categorical variable)}

\begin{itemize}
\tightlist
\item
  The characteristic is a quality.
\item
  The data are categories.
\item
  They cannot be given numerical values.
\item
  However, it may be given a numerical label
\item
  Qualitative variables are sometimes referred as categorical variables.
\item
  \emph{Eg:}
\end{itemize}

\emph{Gender:}

\emph{Age group:}

\emph{Education level:}

\emph{A/L stream:}

\emph{Degree type:}

\emph{Hair colour: }

\emph{FIT student batch:}

\emph{Undergraduate level:}

\emph{Grade that you can obtain for CM 1110/ CM1130}

\textbf{ii Quantitative variable }

\begin{itemize}
\tightlist
\item
  The characteristic is a quantity
\item
  The data are numbers
\item
  Quantitative data require numeric values that indicate how much or how many.
\item
  They are obtained by counting or measuring with some scale
\item
  \emph{Eg: }
\end{itemize}

\emph{Number of family members:}

\emph{Number of emails received per day:}

\emph{Weight of a student:}

\emph{Age:}

\emph{Credit balance in the SIM card:}

\emph{Time remaining in class:}

\emph{Temperature:}

\emph{Marks }

\hypertarget{discrete-continuous-variables}{%
\subsubsection{Discrete/ Continuous Variables}\label{discrete-continuous-variables}}

\begin{itemize}
\tightlist
\item
  Quantitative variables can be classified as either discrete or continuous.
\end{itemize}

\textbf{i Discrete Variables}

\begin{itemize}
\tightlist
\item
  Quantitative
\item
  Usually the data are obtained by counting
\item
  There are impossible values between any two possible values
\item
  \emph{Eg:}
\end{itemize}

\emph{Number of family members:}

\emph{Number of emails received per day:}

\textbf{ii Continuous Variables}

\begin{itemize}
\tightlist
\item
  Quantitative
\item
  Usually, the data are obtained by measuring with a scale
\item
  There are no impossible values between any two possible values.(any value between any two possible values is also a possible value)
\item
  i.e a continuous variable can take any value within a specified range.
\item
  \emph{Eg:}
\end{itemize}

\emph{Weight of a student:}

\emph{Age:}

\emph{Credit balance in the SIM card:}

\emph{Time remaining in class:}

\emph{Temperature:}

\emph{Marks}

\hypertarget{scales-of-measurements}{%
\subsection{Scales of Measurements}\label{scales-of-measurements}}

\begin{figure}[h]

{\centering \includegraphics[width=1\linewidth]{figure/box3-1} 

}

\end{figure}

\begin{itemize}
\tightlist
\item
  There are four levels of measurements called, \textbf{nominal, ordinal, interval and ratio.}
\item
  Each levels has its own rules and restrictions
\item
  Different levels of measurement contains different amount of information with respect to whatever the data are measuring
\end{itemize}

\textbf{i Nominal Scale}

\begin{itemize}
\tightlist
\item
  Qualitative
\item
  No order or ranking in categories.
\item
  These categories have to be mutually exclusive, i.e.~it should not be possible to place an individual or object in more than one category
\item
  A name of a category can be substituted by a number, but it will be mere label and have no numerical meaning
\end{itemize}

\textbf{ii Ordinal Scale}

\begin{itemize}
\tightlist
\item
  Qualitative
\item
  Categories can be ordered or ranked
\item
  A name of a category can be substituted by a number, but such a sequence does not indicate absolute quantities.
\item
  Difference between any two numbers on the scale does not have a numerical meaningful.
\item
  It cannot be assumed that the differences between adjacent numbers on the scale are equal.
\end{itemize}

\textbf{iii Interval Scale}

\begin{itemize}
\tightlist
\item
  Quantitative
\item
  Data can be ordered or ranked
\item
  There is no absolute zero point. Zero is only an arbitrary point with which other values can compare
\item
  Difference between two numbers is a meaningful numerical value
\item
  Ration of two numbers is not a meaningful numerical value.
\end{itemize}

\textbf{iv Ratio Scale}

\begin{itemize}
\tightlist
\item
  Quantitative
\item
  Highest level of measurement
\item
  There exist an absolute zero point (It has a true zero point)
\item
  Ratio between different measurements is meaningful
\end{itemize}

\hypertarget{presentation-of-data}{%
\section{Presentation of Data}\label{presentation-of-data}}

The sinking of the Titanic is one of the most infamous shipwrecks in history.

On April 15, 1912, during her maiden voyage, the widely considered ``unsinkable'' RMS Titanic sank after colliding with an iceberg. Unfortunately, there weren't enough lifeboats for everyone onboard, resulting in the death of 1502 out of 2224 passengers and crew

\footnote{Data source: \url{https://www.kaggle.com/varimp/a-mostly-tidyverse-tour-of-the-titanic}}

Here's a quick summary of our variables:

\begin{longtable}[]{@{}ll@{}}
\toprule
\begin{minipage}[b]{0.49\columnwidth}\raggedright
Variable Name\strut
\end{minipage} & \begin{minipage}[b]{0.45\columnwidth}\raggedright
Description\strut
\end{minipage}\tabularnewline
\midrule
\endhead
\begin{minipage}[t]{0.49\columnwidth}\raggedright
PassengerID\strut
\end{minipage} & \begin{minipage}[t]{0.45\columnwidth}\raggedright
Passenger ID (just a row number, so obviously not useful for prediction)\strut
\end{minipage}\tabularnewline
\begin{minipage}[t]{0.49\columnwidth}\raggedright
Survived\strut
\end{minipage} & \begin{minipage}[t]{0.45\columnwidth}\raggedright
Survived (1) or died (0)\strut
\end{minipage}\tabularnewline
\begin{minipage}[t]{0.49\columnwidth}\raggedright
Pclass\strut
\end{minipage} & \begin{minipage}[t]{0.45\columnwidth}\raggedright
Passenger class (first, second or third)\strut
\end{minipage}\tabularnewline
\begin{minipage}[t]{0.49\columnwidth}\raggedright
Name\strut
\end{minipage} & \begin{minipage}[t]{0.45\columnwidth}\raggedright
Passenger name\strut
\end{minipage}\tabularnewline
\begin{minipage}[t]{0.49\columnwidth}\raggedright
Gender\strut
\end{minipage} & \begin{minipage}[t]{0.45\columnwidth}\raggedright
Passenger Gender\strut
\end{minipage}\tabularnewline
\begin{minipage}[t]{0.49\columnwidth}\raggedright
Age\strut
\end{minipage} & \begin{minipage}[t]{0.45\columnwidth}\raggedright
Passenger age\strut
\end{minipage}\tabularnewline
\begin{minipage}[t]{0.49\columnwidth}\raggedright
SibSp\strut
\end{minipage} & \begin{minipage}[t]{0.45\columnwidth}\raggedright
Number of siblings/spouses aboard\strut
\end{minipage}\tabularnewline
\begin{minipage}[t]{0.49\columnwidth}\raggedright
Parch\strut
\end{minipage} & \begin{minipage}[t]{0.45\columnwidth}\raggedright
Number of parents/children aboard\strut
\end{minipage}\tabularnewline
\begin{minipage}[t]{0.49\columnwidth}\raggedright
Ticket\strut
\end{minipage} & \begin{minipage}[t]{0.45\columnwidth}\raggedright
Ticket number\strut
\end{minipage}\tabularnewline
\begin{minipage}[t]{0.49\columnwidth}\raggedright
Fare\strut
\end{minipage} & \begin{minipage}[t]{0.45\columnwidth}\raggedright
Fare\strut
\end{minipage}\tabularnewline
\begin{minipage}[t]{0.49\columnwidth}\raggedright
Cabin\strut
\end{minipage} & \begin{minipage}[t]{0.45\columnwidth}\raggedright
Cabin\strut
\end{minipage}\tabularnewline
\begin{minipage}[t]{0.49\columnwidth}\raggedright
Embarked\strut
\end{minipage} & \begin{minipage}[t]{0.45\columnwidth}\raggedright
Port of embarkation (S = Southampton, C = Cherbourg, Q = Queenstown)\strut
\end{minipage}\tabularnewline
\bottomrule
\end{longtable}

\hypertarget{tabular-presentations-of-data}{%
\subsection{Tabular Presentations of Data}\label{tabular-presentations-of-data}}

\textbf{Raw Data}

\begin{itemize}
\tightlist
\item
  Raw data are collected data that have not been organized numerically
\item
  Eg: Passenger age
\end{itemize}

\begin{verbatim}
##   PassengerId Pclass                                         Name    Sex  Age
## 1         892      3                             Kelly, Mr. James   male 34.5
## 2         893      3             Wilkes, Mrs. James (Ellen Needs) female 47.0
## 3         894      2                    Myles, Mr. Thomas Francis   male 62.0
## 4         895      3                             Wirz, Mr. Albert   male 27.0
## 5         896      3 Hirvonen, Mrs. Alexander (Helga E Lindqvist) female 22.0
## 6         897      3                   Svensson, Mr. Johan Cervin   male 14.0
##   SibSp Parch  Ticket    Fare Cabin Embarked
## 1     0     0  330911  7.8292              Q
## 2     1     0  363272  7.0000              S
## 3     0     0  240276  9.6875              Q
## 4     0     0  315154  8.6625              S
## 5     1     1 3101298 12.2875              S
## 6     0     0    7538  9.2250              S
\end{verbatim}

\begin{verbatim}
##  [1] 34.5 47.0 62.0 27.0 22.0 14.0 30.0 26.0 18.0 21.0   NA 46.0 23.0 63.0 47.0
## [16] 24.0 35.0 21.0 27.0 45.0 55.0  9.0   NA 21.0 48.0 50.0 22.0 22.5 41.0   NA
## [31] 50.0 24.0 33.0   NA 30.0 18.5   NA 21.0 25.0   NA
\end{verbatim}

\textbf{An array}

\begin{itemize}
\tightlist
\item
  An array is an arrangement of raw numerical data in ascending or descending order of magnitude.
\item
  Eg: Passenger age
\end{itemize}

\begin{verbatim}
##  [1]  9.0 14.0 18.0 18.5 21.0 21.0 21.0 21.0 22.0 22.0 22.5 23.0 24.0 24.0 25.0
## [16] 26.0 27.0 27.0 30.0 30.0 33.0 34.5 35.0 41.0 45.0 46.0 47.0 47.0 48.0 50.0
## [31] 50.0 55.0 62.0 63.0
\end{verbatim}

\textbf{Frequency Table (Frequency Distributions)}

\begin{itemize}
\tightlist
\item
  A frequency table (frequency distribution) is a listing of the values a variable takes in a data set, along with how often (frequently) each value occurs
\item
  frequency can be recorded as a

  \begin{itemize}
  \tightlist
  \item
    frequency or count: the number of times a value occurs, or
  \item
    percentage frequency: the percentage of times a value occurs
  \end{itemize}
\item
  Percentage frequency can be calculated as,
\end{itemize}

\hypertarget{sets-and-relations}{%
\chapter{Sets and Relations}\label{sets-and-relations}}

\hypertarget{probability}{%
\chapter{Probability}\label{probability}}

\hypertarget{correlation-and-regression}{%
\chapter{Correlation and Regression}\label{correlation-and-regression}}

  \bibliography{book.bib,packages.bib}

\end{document}
